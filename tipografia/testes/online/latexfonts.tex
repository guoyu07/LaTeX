\documentclass[a4paper]{article}
\usepackage[utf8]{inputenc}
\usepackage[T1]{fontenc}
\usepackage[brazil]{babel}
\usepackage{hyperref}
\usepackage{lipsum}
\usepackage{anttor}
\usepackage[T1]{pbsi} %brushscr
\usepackage{calligra}
\usepackage{concrete}
\usepackage[clock,weather]{ifsym}
\usepackage[math]{iwona}
\usepackage{yfonts}
\usepackage[margin=3cm]{geometry}

\newcommand{\frase}{Tem Coelhos que fogem de Abelhas Brigando por Jabuti porque faz chover uva e xadrez}
\renewcommand\familydefault{cmr}

%familia
\newcommand{\fonte}[2]{
  {\fontfamily{#1}\selectfont #2}
}

\pagestyle{empty}
\begin{document}

Fontes descritas em \href{http://suppiya.files.wordpress.com/2008/02/latex_fonts.pdf}{LaTeX Fonts} de Ki-Joo Kim.

\subsection*{Computer Modern Roman (padrão)}

\frase

\subsection*{antt}

\verb|\usepackage{anttor}|

\fonte{antt}{\frase}

\emph{\fonte{antt}{\frase}}


\subsection*{augie}

Não precisa carregar pacote.

\fonte{augie}{\frase}


\subsection*{brushscr}

\verb|\usepackage[T1]{pbsi}|

\textbsi{\frase}


\subsection*{calligra}

\verb|\usepackage{calligra}|

\fonte{calligra}{\frase}


\subsection*{concrete}

\verb|\usepackage{concrete}|

\fonte{ccr}{\frase}

\emph{\fonte{ccr}{\frase}}


\subsection*{ifsym}

\verb|\usepackage[clock,weather]{ifsym}|

O pacote \texttt{ifsym} é um pacote de símbolos, dentre eles: \emph{clock} que mostra as horas: \verb|\showclock{3}{40}| \showclock{3}{40}.

\verb|\SunCloud| que mostra um sol com nuvem \SunCloud{} e \verb|\textifsym{123.45}| que mostra \textifsym{123.45}.

Para ver os demais símbolos digite \verb|texdoc ifsym| no terminal.


\subsection*{iwona}

\verb|\usepackage[math]{iwona}|

Este pacote possui suporte para modo matemático.

\fonte{iwona}{\frase}

\emph{\fonte{iwona}{\frase}}

\fonte{iwona}{$1 2 3 4 5 6 7 8 9 0$} - modo matemático.
 
\fonte{iwona}{1 2 3 4 5 6 7 8 9 0} - modo texto.


\subsection*{yfonts}

\verb|\usepackage{yfonts}|

\textbf{Gotisch}

\textgoth{\frase}

\textbf{Schwabacher}

\textswab{\frase}

\textbf{Fraktur}

\textfrak{\frase}

\textbf{Baroque}

\textinit{A B C D E}

\yinipar{L}etra grande ao iniciar uma frase. Uma frase grande com pelo menos 4 linhas de texto. Este tipo de fonte era usado em livros antigos, e até hoje ainda é usado em livros sagrados ou clássicos. Esta fonte dá um estilo decorativo a frase. Para escrevê-lo digite, por exemplo, \verb|\yinipar{L}etra|. Se o parágrafo tiver várias linhas de texto a letra quebra as linhas iniciais do texto.


\end{document}